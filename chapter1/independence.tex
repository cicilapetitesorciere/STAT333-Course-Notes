\section{Statistical Independence}
\subsection{Independence with two Events}
Two Events $A$ and $B$ are \textbf{Independent} if they do not have any ``information" about the other. In other words knowing that $B$ has occurred does not change the probability of $A$ having occurred. We can express this symbolically as follows
\[
    P(A|B)=P(A)
\]
or equivalently:
\[
    P(A\cap B)=P(A)P(B)
\]
\begin{notsofast}
    Independence is \emph{not} the same as Mutual Exclusivity. In fact if $A$ and $B$ are Mutually Exclusive then they cannot be independent. 
    \begin{proof}
        Suppose $A$ and $B$ are two Mutually Exclusive Events and we know that $B$ has occurred. Then we also know $A$ must not have occurred. In other words, knowing $B$ has occurred has given us information about the probability of $A$ (namely that the probability is zero). 
    \end{proof}
\end{notsofast}

\begin{lemma}
Let $A$ and $B$ be Independent Events. Then the following pairs are also Independent:
\begin{enumerate}
    \item $A$ and $\bar B$
    \item $\bar A$ and $B$
    \item $\bar A$ and $\bar B$
\end{enumerate}

\begin{proof}
    \begin{enumerate}
        \item[]
        \item\todo
        \item\todo
        \item \todo
    \end{enumerate}
\end{proof}
\end{lemma}

\subsection{Independence with more than two Events}

\subsection{Conditional Independence}


Two events are \textbf{Conditionally Independent} given $C$ if 
\[
    P(A\cap B | C)= P(A|C)P(B|C)
\]

\todo




