
\section{The Secretary Problem}
Suppose you have applied to three jobs and are waiting to hear back. You are pretty sure that all three companies will give you an offer, but you do not know what salary they will offer you. And after each company presents their offer, you will have to decide whether to accept or reject it, and will not have time wait around to see what the next company is planning to offer you. What is the optimal strategy for maximizing your chances of accepting the best offer?
\\\\ 
As it turns out, the best strategy is to reject the first regardless of what they offer, and then accept any offer better than that. To illustrate this, let us suppose we have three companies $A$ (best offer), $B$ (second-best offer), and $C$ (worst offer). There are $3!=6$ orders in which they may call you. We will assume that all orders are equally likely, and consider what happens in each scenario:
\\\\
\allowdisplaybreaks
\begin{tabularx}{325pt}{cXc}

1.\ \textbf{ABC:} 
&
So far, our strategy is not off to a great start. We reject $A$ and are left disappointed by the remaining two offers. 
&
\raisebox{-0.75\height}{\includesvg[width=25px]{symbols/xcheck-red-x.svg}}
\\\\
2.\ \textbf{BAC:} 
&
This works out better a bit better. We reject $B$ and and then get the better offer $A$, which turns out to be the best.
&
\raisebox{-0.75\height}{\includesvg[width=25px]{symbols/xcheck-green-check.svg}}
\\\\
3.\ \textbf{BCA:} 
&
Here we reject $B$ and then $C$, and finally accept the best offer $A$.
&
\raisebox{-0.75\height}{\includesvg[width=25px]{symbols/xcheck-green-check.svg}}
\\\\
4.\ \textbf{CBA:} 
&
Here we do a little worse. By starting with $C$ we end up setting our standards too low and accept $B$ instead of holding out for the better offer
&
\raisebox{-0.75\height}{\includesvg[width=25px]{symbols/xcheck-red-x.svg}}
\\\\

5.\ \textbf{CAB:} 
&
But here, those low standards don't end up mattering, since the best offer happens to be the very next one.
&
\raisebox{-0.75\height}{\includesvg[width=25px]{symbols/xcheck-green-check.svg}}
\\\\

6.\ \textbf{ACB:} 
&
And once again, we end up the same problem we encountered in the first round, setting our standards too high and missing out on our best offer.
&
\raisebox{-0.75\height}{\includesvg[width=25px]{symbols/xcheck-red-x.svg}}

\end{tabularx}
\\\\

All in all, we end up getting the best offer half of the time.

\todo JUSTIFY THE GENERALIZATION

However let's generalize this this to $n$ people. (Answer $\nicefrac 1e$). This is often known as the 37\% rule. 
